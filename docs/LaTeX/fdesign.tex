\documentclass[fontsize=9pt, parskip=half, notitlepage, fleqn]{scrartcl}

\usepackage{tabularx}
\usepackage{booktabs}
\usepackage{verbatim}

\usepackage[UKenglish]{babel}
\usepackage[utf8]{inputenc}
\usepackage{lmodern}
\usepackage[T1]{fontenc}
\usepackage{amssymb, amsmath, amsfonts, amsthm, empheq}
\usepackage[dvipsnames]{xcolor}
\usepackage{xspace}
\usepackage{siunitx}
\usepackage[margin=3cm]{geometry}
\usepackage{natbib}
\usepackage[pdftex,final,allcolors=myblue,colorlinks,breaklinks=True]{hyperref}

\frenchspacing
\definecolor{myblue}{rgb}{0,0,.8} % {0,0,.8}
\newcommand{\mr}[1]{\mathrm{#1}}

\title{Digital Linear Filter (DLF) design}
\author{Dieter Werthmüller}
\subtitle{Some notes regarding the \texttt{fdesign} add-on for \texttt{empymod}}

\begin{document}
\maketitle

\section{About and Info}

The add-on \texttt{fdesign} can be used to design digital linear filters for
the Hankel or Fourier transform, or for any linear transform. For this included
or provided theoretical transform pairs can be used. Alternatively, one can use
the EM modeller \texttt{empymod} \citep{GEO.17.Werthmuller} to use the
responses to an arbitrary 1D model as numerical transform pair.

More information can be found in the following places:

\begin{itemize}
  \item The article about \texttt{fdesign} is in the repo
    \href{https://github.com/empymod/article-fdesign}{github.com/empymod/article-fdesign}
  \item Example notebooks to design a filter can be found in the repo
    \href{https://github.com/empymod/example-notebooks}{github.com/empymod/example-notebooks}
\end{itemize}


The methodology of \texttt{fdesign} is based upon \cite{GP.07.Kong}. The whole
project of \texttt{fdesign} started with the Matlab scripts from Kerry Key,
which he used to design his filters for \cite{GEO.09.Key, GEO.12.Key}. Fruitful
discussions with Evert Slob and Kerry Key improved the add-on substantially.

Note that the use of \texttt{empymod} to create numerical transform pairs is,
as of now, only implemented for the Hankel transform (via
\texttt{fdesign.empy\_hankel}).

\section{Implemented analytical transform pairs}

The following tables list the transform pairs which are implemented by default.
Any other transform pair can be provided as input. A transform pair is defined
in the following way:
\begin{verbatim}
    from empymod.scripts.fdesign import Ghosh

    def my_tp_pair(var):
        """My transform pair."""

        def lhs(l):
            return func(l, var)

        def rhs(r):
            return func(r, var)

        return Ghosh(name, lhs, rhs)
\end{verbatim}

Here, \texttt{name} must be one of \texttt{j0}, \texttt{j1}, \texttt{sin}, or
\texttt{cos}, depending what type of transform pair it is. Additional variables
are provided with \texttt{var}. The evaluation points of the \texttt{lhs} are
denoted by \texttt{l}, and the evaluation points of the \texttt{rhs} are
denoted as \texttt{r}. As an example here the implemented transform pair
\texttt{j0\_1}:
\begin{verbatim}
    def j0_1(a=1):
        """Hankel transform pair J0_1 ([Anderson_1975]_)."""

        def lhs(l):
            return l*np.exp(-a*l**2)

        def rhs(r):
            return np.exp(-r**2/(4*a))/(2*a)

        return Ghosh('j0', lhs, rhs)
\end{verbatim}

\subsection{Implemented Hankel transforms}

\renewcommand{\arraystretch}{1.3}
\begin{tabularx}{\linewidth}{lp{2.6cm}X}
  Name & Reference & Transform pair \\
\toprule
   %
  j0\_1& \cite{USGS.75.Anderson}&
  \noindent\parbox[c]{\hsize}{
  \begin{equation}
    \int^\infty_0\,l \exp\left(-al^2\right) J_0(lr)\,\mr{d}l =
    \frac{\exp\left(\frac{-r^2}{4a}\right)}{2a}
    \label{eq:j0_1}
  \end{equation}
  } \\
   %
  j0\_2& \cite{USGS.75.Anderson}&
  \noindent\parbox[c]{\hsize}{
  \begin{equation}
    \int^\infty_0\,\exp\left(-al\right) J_0(lr)\,\mr{d}l =
    \frac{1}{\sqrt{a^2+r^2}}
    \label{eq:j0_2}
  \end{equation}
  } \\
   %
  j0\_3&
  \noindent\parbox[c]{\hsize}{\cite{GP.97.Guptasarma}}&
  \noindent\parbox[c]{\hsize}{
  \begin{equation}
    \int^\infty_0\,l\exp\left(-al\right) J_0(lr)\,\mr{d}l =
    \frac{a}{(a^2 + r^2)^{3/2}}
    \label{eq:j0_3}
  \end{equation}
  } \\
   %
  j0\_4&
  \noindent\parbox[c]{\hsize}{\cite{JGR.82.Chave}}&
  \noindent\parbox[c]{\hsize}{
  \begin{equation}
    \int^\infty_0\,\frac{l}{\beta} \exp\left(-\beta z_\mr{v} \right)
    J_0(lr)\,\mr{d}l =
    \frac{\exp\left(-\gamma R\right)}{R}
    \label{eq:j0_4}
  \end{equation}
  } \\
   %
  j0\_5&
  \noindent\parbox[c]{\hsize}{\cite{JGR.82.Chave}}&
  \noindent\parbox[c]{\hsize}{
  \begin{equation}
    \int^\infty_0\,l \exp\left(-\beta z_\mr{v} \right)
    J_0(lr)\,\mr{d}l =
    \frac{ z_\mr{v} (\gamma R + 1)}{R^3}\exp\left(-\gamma R\right)
    \label{eq:j0_4}
  \end{equation}
  } \\
   %
  j1\_1& \cite{USGS.75.Anderson}&
  \noindent\parbox[c]{\hsize}{
  \begin{equation}
    \int^\infty_0\,l^2 \exp\left(-al^2\right) J_1(lr)\,\mr{d}l =
    \frac{r}{4a^2} \exp\left(-\frac{r^2}{4a}\right)
    \label{eq:j1_1}
  \end{equation}
  } \\
   %
  j1\_2& \cite{USGS.75.Anderson}&
  \noindent\parbox[c]{\hsize}{
  \begin{equation}
    \int^\infty_0\,\exp\left(-al\right) J_1(lr)\,\mr{d}l =
    \frac{\sqrt{a^2+r^2}-a}{r\sqrt{a^2 + r^2}}
    \label{eq:j1_2}
  \end{equation}
  } \\
   %
  j1\_3& \cite{USGS.75.Anderson}&
  \noindent\parbox[c]{\hsize}{
  \begin{equation}
    \int^\infty_0\,l \exp\left(-al\right) J_1(lr)\,\mr{d}l =
    \frac{r}{(a^2 + r^2)^{3/2}}
    \label{eq:j1_3}
  \end{equation}
  } \\
   %
  j1\_4&
  \noindent\parbox[c]{\hsize}{\cite{JGR.82.Chave}}&
  \noindent\parbox[c]{\hsize}{
  \begin{equation}
    \int^\infty_0\,\frac{l^2}{\beta} \exp\left(-\beta z_\mr{v} \right)
    J_1(lr)\,\mr{d}l =
    \frac{r(\gamma R+1)}{R^3}\exp\left(-\gamma R\right)
    \label{eq:j1_4}
  \end{equation}
  } \\
   %
  j1\_5&
  \noindent\parbox[c]{\hsize}{\cite{JGR.82.Chave}}&
  \noindent\parbox[c]{\hsize}{
  \begin{equation}
    \int^\infty_0\,l^2 \exp\left(-\beta z_\mr{v} \right)
    J_1(lr)\,\mr{d}l =
    \frac{r z_\mr{v} (\gamma^2R^2+3\gamma R+3)}{R^5}\exp\left(-\gamma R\right)
    \label{eq:j1_4}
  \end{equation}
  } \\
\end{tabularx}

\begin{align}
  a&>0,\quad r>0\\
  %
  z_\mr{v} &= |z_\mr{rec} - z_\mr{src}|\\
%
  R &= \sqrt{r^2 + z_\mr{v}^2}\\
%
    \gamma &= \sqrt{2j\pi\mu_0f/\rho}\\
%
    \beta &= \sqrt{l^2 + \gamma^2}\\
  \label{eq:symb}
\end{align}


\subsection{Implemented Fourier transforms}

\renewcommand{\arraystretch}{1.3}
\begin{tabularx}{\linewidth}{llX}
  Name & Reference & Transform pair \\
\toprule
   %
  sin\_1& \cite{USGS.75.Anderson}&
  \noindent\parbox[c]{\hsize}{
  \begin{equation}
    \int^\infty_0\,l\exp\left(-a^2l^2\right) \sin(lr)\,\mr{d}l =
    \frac{\sqrt{\pi}r}{4a^3} \exp\left(-\frac{r^2}{4a^2}\right)
    \label{eq:sin_1}
  \end{equation}
  } \\
   %
  sin\_2& \cite{USGS.75.Anderson}&
  \noindent\parbox[c]{\hsize}{
  \begin{equation}
    \int^\infty_0\,\exp\left(-al\right) \sin(lr)\,\mr{d}l =
    \frac{r}{a^2 + r^2}
    \label{eq:sin_2}
  \end{equation}
  } \\
   %
  sin\_3& \cite{USGS.75.Anderson}&
  \noindent\parbox[c]{\hsize}{
  \begin{equation}
    \int^\infty_0\,\frac{l}{a^2+l^2} \sin(lr)\,\mr{d}l =
    \frac{\pi}{2} \exp\left(-ar\right)
    \label{eq:sin_3}
  \end{equation}
  } \\
   %
  cos\_1& \cite{USGS.75.Anderson}&
  \noindent\parbox[c]{\hsize}{
  \begin{equation}
    \int^\infty_0\,\exp\left(-a^2l^2\right) \cos(lr)\,\mr{d}l =
    \frac{\sqrt{\pi}}{2a} \exp\left(-\frac{r^2}{4a^2}\right)
    \label{eq:cos_1}
  \end{equation}
  } \\
   %
  cos\_2& \cite{USGS.75.Anderson}&
  \noindent\parbox[c]{\hsize}{
  \begin{equation}
    \int^\infty_0\,\exp\left(-al\right) \cos(lr)\,\mr{d}l =
    \frac{a}{a^2 + r^2}
    \label{eq:cos_2}
  \end{equation}
  } \\
   %
  cos\_3& \cite{USGS.75.Anderson}&
  \noindent\parbox[c]{\hsize}{
  \begin{equation}
    \int^\infty_0\,\frac{1}{a^2+l^2} \cos(lr)\,\mr{d}l =
    \frac{\pi}{2a} \exp\left(-ar\right)
    \label{eq:cos_3}
  \end{equation}
  } \\
   %
\end{tabularx}

%% ~ REFERENCES
\begin{thebibliography}{}
\itemsep0pt

\bibitem[Anderson, 1975]{USGS.75.Anderson}
Anderson, W.~L.,  1975, Improved digital filters for evaluating {F}ourier and
  {H}ankel transform integrals:
\newblock USGS, {\bf PB242800};
  \href{https://pubs.er.usgs.gov/publication/70045426}{https://pubs.er.usgs.gov/publication/70045426}.

\bibitem[Chave and Cox, 1982]{JGR.82.Chave}
Chave, A.~D., and C.~S. Cox,  1982, Controlled electromagnetic sources for
  measuring electrical conductivity beneath the oceans: 1. forward problem and
  model study: Journal of Geophysical Research,
\newblock {\bf 87}, 5327--5338;
  \href{http://doi.org/10.1029/JB087iB07p05327}{doi: 10.1029/JB087iB07p05327}.

\bibitem[Guptasarma and Singh, 1997]{GP.97.Guptasarma}
Guptasarma, D., and B. Singh,  1997, New digital linear filters for {H}ankel
  {J}0 and {J}1 transforms: Geophysical Prospecting,
\newblock {\bf 45}, 745--762;
  \href{http://doi.org/10.1046/j.1365-2478.1997.500292.x}{doi:
  10.1046/10.1046/j.1365-2478.1997.500292.x}.

\bibitem[Kong, 2007]{GP.07.Kong}
Kong, F.~N.,  2007, Hankel transform filters for dipole antenna radiation in a
  conductive medium: Geophysical Prospecting, {\bf 55}, 83--89.
\newblock (\href{http://doi.org/10.1111/j.1365-2478.2006.00585.x}{doi:
  10.1111/j.1365-2478.2006.00585.x}).

\bibitem[Key, 2009]{GEO.09.Key}
Key, K.,  2009, {1D} inversion of multicomponent, multifrequency marine {CSEM}
  data: {M}ethodology and synthetic studies for resolving thin resistive
  layers: Geophysics, {\bf 74}, F9--F20.
\newblock (\href{http://doi.org/10.1190/1.3058434}{doi: 10.1190/1.3058434}).

\bibitem[Key, 2012]{GEO.12.Key}
--------, 2012, Is the fast {H}ankel transform faster than quadrature?:
  Geophysics, {\bf 77}, F21--F30.
\newblock (\href{http://doi.org/10.1190/GEO2011-0237.1}{doi:
  10.1190/GEO2011-0237.1}).

\bibitem[Werthmüller, 2017]{GEO.17.Werthmuller}
Werthmüller, D.,  2017, An open-source full {3D} electromagnetic modeler for
  {1D} {VTI} media in {P}ython: empymod: Geophysics, {\bf 82}, WB9--WB19..
\newblock (\href{http://doi.org/10.1190/geo2016-0626.1}{doi:
  10.1190/geo2016-0626.1}).


\end{thebibliography}

\end{document}

