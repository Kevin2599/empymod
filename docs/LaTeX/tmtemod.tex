\documentclass[fontsize=9pt, parskip=half, notitlepage, fleqn]{scrartcl}

\usepackage[UKenglish]{babel}
\usepackage[utf8]{inputenc}
\usepackage{lmodern}
\usepackage[T1]{fontenc}
\usepackage{amssymb, amsmath, amsfonts, amsthm, empheq}
\usepackage[dvipsnames]{xcolor}
\usepackage{xspace}
\usepackage{siunitx}
\usepackage[margin=3cm]{geometry}
\usepackage{natbib}

\frenchspacing

% So equation numbers are not coloured
\makeatletter
\renewcommand\tagform@[1]{%
   \maketag@@@{\normalcolor\ignorespaces(#1)\unskip\@@italiccorr}}
\makeatother

\newcommand{\bkcol}[1]{\textcolor{MidnightBlue}{#1\xspace}}
\newcommand{\hun}{\textsc{Hun15}\xspace}
\newcommand{\book}{\bkcol{\textsc{Book}}\xspace}
\newcommand{\empymod}{\texttt{empymod}\xspace}
\newcommand{\tmtemod}{\texttt{tmtemod.py}\xspace}
\newcommand{\mr}[1]{\mathrm{#1}}

\begin{document}
\setlength{\jot}{10pt}  % Increase spacing between equations

\section*{Adjust Hunziker et al. (2015) for TM/TE-split}

{\ttfamily \small Version 1.0, \today

\hun refers to \cite{GEO.15.Hunziker}, \book refers to the derivation of
\cite{CUP.17.Ziolkowski}.}

The modeller \empymod returns the total field, hence not distinguishing between
TM and TE mode, and even less between up- and down-going fields. The reason
behind this is simple: The derivation of \hun, on which \empymod is based,
returns the total field. Internally it also calculates TM and TE modes, and
sums these up. However, the separation into TM and TE mode introduces a
singularity at $\kappa = 0$. It has no contribution in the space-frequency
domain to the total fields, but it introduces non-physical events in each mode
with opposite sign (so they cancel each other out in the total field). In
order to obtain the correct TM and TE contributions one has to remove these
non-physical parts.

To remove the non-physical part we use the file \tmtemod in this
directory. This routine is basically a heavily simplified version of \empymod
with the following limitations:
\begin{itemize}
  \item x-directed electric sources and electric receivers;
  \item only frequency domain;
  \item direct field is always calculated in the wavenumber-domain;
  \item the Fast Hankel transform is used with a 201\,pt filter;
  \item source and receivers have to be in the same layer;
  \item the model must have more than one layer (there is only direct field
    contribution anyway for a fullspace); and
  \item electric permittivity and magnetic permeability are isotropic.
\end{itemize}

So \tmtemod returns the signal separated into TM$^{++}$, TM$^{+-}$, TM$^{-+}$,
TM$^{--}$, TE$^{++}$, TE$^{+-}$, TE$^{-+}$, and TE$^{--}$ as well as the direct
field TM and TE contributions. The first superscript denotes the direction in
which the field diffuses towards the receiver and the second superscript
denotes the direction in which the field diffuses away from the source. For
both the plus-sign indicates the field diffuses in the downward direction and
the minus-sign indicates the field diffuses in the upward direction. The
routine uses \empymod wherever possible, see the corresponding functions in
\empymod for more explanation and documentation regarding input parameters.

We start with equation (105) in \hun:
%
\begin{align}
  \hat{G}^{ee}_{xx}(\boldsymbol{x}, \boldsymbol{x'}, \omega)& =
  \hat{G}^{ee;i}_{xx;s}(\boldsymbol{x}-\boldsymbol{x'}, \omega)
  + \frac{1}{8\pi}\int^\infty_{\kappa=0}
  \left(\frac{\Gamma_s \tilde{g}^{tm}_{hh;s}}{\eta_s}-
  \frac{\zeta_s \tilde{g}^{te}_{zz;s}}{\bar{\Gamma}_s}\right)
  J_0(\kappa r)\kappa\,\mr{d}\kappa \nonumber\\
  %
  &\quad - \frac{\cos(2\phi)}{8\pi}\int^\infty_{\kappa=0}
  \left(\frac{\Gamma_s \tilde{g}^{tm}_{hh;s}}{\eta_s} +
  \frac{\zeta_s \tilde{g}^{te}_{zz;s}}{\bar{\Gamma}_s}\right)
  J_2(\kappa r)\kappa\,\mr{d}\kappa \ .
\end{align}
%

Ignoring the incident field, and using $J_2 = \frac{2}{\kappa r}J_1 - J_0$ to
avoid $J_2$-integrals, we get
%
\begin{align}
  \hat{G}^{ee}_{xx}(\boldsymbol{x}, \boldsymbol{x'}, \omega)& =
  \frac{1}{8\pi}\int^\infty_{\kappa=0}
  \left(\frac{\Gamma_s \tilde{g}^{tm}_{hh;s}}{\eta_s}-
  \frac{\zeta_s \tilde{g}^{te}_{zz;s}}{\bar{\Gamma}_s}\right)
  J_0(\kappa r)\kappa\,\mr{d}\kappa \nonumber\\
  %
  &\quad + \frac{\cos(2\phi)}{8\pi}\int^\infty_{\kappa=0}
  \left(\frac{\Gamma_s \tilde{g}^{tm}_{hh;s}}{\eta_s} +
  \frac{\zeta_s \tilde{g}^{te}_{zz;s}}{\bar{\Gamma}_s}\right)
  J_0(\kappa r)\kappa\,\mr{d}\kappa \nonumber\\
  %
  &\quad - \frac{\cos(2\phi)}{4\pi r}\int^\infty_{\kappa=0}
  \left(\frac{\Gamma_s \tilde{g}^{tm}_{hh;s}}{\eta_s} +
  \frac{\zeta_s \tilde{g}^{te}_{zz;s}}{\bar{\Gamma}_s}\right)
  J_1(\kappa r)\,\mr{d}\kappa \ .
\end{align}
%

From this the TM- and TE-parts follow as
%
\begin{align}
  \mr{TE}& = \frac{\cos(2\phi)-1}{8\pi}\int^\infty_{\kappa=0}
  \frac{\zeta_s \tilde{g}^{te}_{zz;s}}{\bar{\Gamma}_s}
  J_0(\kappa r)\kappa\,\mr{d}\kappa 
   - \frac{\cos(2\phi)}{4\pi r}\int^\infty_{\kappa=0}
  \frac{\zeta_s \tilde{g}^{te}_{zz;s}}{\bar{\Gamma}_s}
  J_1(\kappa r)\,\mr{d}\kappa \ , \\
  %
  \mr{TM}& = \frac{\cos(2\phi)+1}{8\pi}\int^\infty_{\kappa=0}
  \frac{\Gamma_s \tilde{g}^{tm}_{hh;s}}{\eta_s}
  J_0(\kappa r)\kappa\,\mr{d}\kappa
  - \frac{\cos(2\phi)}{4\pi r}\int^\infty_{\kappa=0}
  \frac{\Gamma_s \tilde{g}^{tm}_{hh;s}}{\eta_s}
  J_1(\kappa r)\,\mr{d}\kappa \ .
\end{align}
%

Equations (108) and (109) in \hun yield the required parameters
$\tilde{g}^{tm}_{hh;s}$ and $\tilde{g}^{te}_{zz;s}$,
%
\begin{align}
  \tilde{g}^{tm}_{hh;s}& = P^{u-}_s W^u_s + P^{d-}_s W^d_s \ , \\
  \tilde{g}^{te}_{zz;s}& = \bar{P}^{u+}_s \bar{W}^u_s +
                           \bar{P}^{d+}_s \bar{W}^d_s \ .
\end{align}
%

The parameters $P^{u\pm}_s$ and $P^{d\pm}_s$ are given in equations (81) and
(82), $\bar{P}^{u\pm}_s$ and $\bar{P}^{d\pm}_s$ in equations (A-8) and (A-9);
$W^u_s$ and $W^d_s$ in equation (74). This yields
%
\begin{align}
  \tilde{g}^{te}_{zz;s} &=
  %
  \frac{\bar{R}_s^+}{\bar{M}_s}\left\{\exp[-\bar{\Gamma}_s(z_s-z+d^+)] +
    \bar{R}_s^-\exp[-\bar{\Gamma}_s(z_s-z+d_s+d^-)]\right\} \nonumber \\
  %
  &\quad +
    \frac{\bar{R}_s^-}{\bar{M}_s}\left\{\exp[-\bar{\Gamma}_s(z-z_{s-1}+d^-)]+
    \bar{R}_s^+\exp[-\bar{\Gamma}_s(z-z_{s-1}+d_s+d^+)]\right\}\nonumber \ ,\\
  %
  &=\frac{\bar{R}_s^+}{\bar{M}_s}\left\{\exp[-\bar{\Gamma}_s(2z_s-z-z')] +
    \bar{R}_s^-\exp[-\bar{\Gamma}_s(z'-z+2d_s)]\right\} \nonumber \\
  %
  &\quad +
  \frac{\bar{R}_s^-}{\bar{M}_s}\left\{\exp[-\bar{\Gamma}_s(z+z'-2z_{s-1})]+
  \bar{R}_s^+\exp[-\bar{\Gamma}_s(z-z'+2d_s)]\right\}\ ,
  %
\end{align}
%
where $d^\pm$ is taken from the text below equation (67). There are four terms
in the right-hand side, two in the first line and two in the second line. The
first term in the first line is the integrand of TE$^{+-}$, the second term in
the first line corresponds to TE$^{++}$, the first term in the second line is
TE$^{-+}$, and the second term in the second line is TE$^{--}$.

If we look at TE$^{+-}$, we have
%
\begin{equation}
  \tilde{g}^{te+-}_{zz;s} =
  \frac{\bar{R}_s^+}{\bar{M}_s}\exp[-\bar{\Gamma}_s(2z_s-z-z')] \ ,
  %
\end{equation}
%
and therefore
%
\begin{align}
  \mr{TE}^{+-}& = \frac{\cos(2\phi)-1}{8\pi}\int^\infty_{\kappa=0}
  \frac{\zeta_s \bar{R}_s^+}{\bar{\Gamma}_s\bar{M}_s}
  \exp[-\bar{\Gamma}_s(2z_s-z-z')]
  J_0(\kappa r)\kappa\,\mr{d}\kappa \nonumber \\
  &\quad - \frac{\cos(2\phi)}{4\pi r}\int^\infty_{\kappa=0}
  \frac{\zeta_s \bar{R}_s^+}{\bar{\Gamma}_s\bar{M}_s}
  \exp[-\bar{\Gamma}_s(2z_s-z-z')]
  J_1(\kappa r)\,\mr{d}\kappa \ .
  \label{eq:hunte}
\end{align}
%

We can compare this to equation (4.165) in \book, with $\hat{I}^e_x=1$ and
slightly re-arranging it to look more alike, we get
%
\bkcol{
\begin{align}
  \hat{E}^{+-}_{xx;H} &= \frac{y^2}{4\pi r^2}
  \int^\infty_{\kappa=0} \frac{\zeta_1}{\Gamma_1} 
  \frac{R^-_{H;1}}{M_{H;1}}
  \exp(-\Gamma_1 h^{+-})J_0(\kappa r)\kappa\rm{d}\kappa \nonumber \\
  %
  &\quad + \frac{x^2-y^2}{4\pi r^3}
  \int^\infty_{\kappa=0} \frac{\zeta_1}{\Gamma_1} 
  \left(\frac{R^-_{H;1}}{M_{H;1}} -
  \frac{R^-_{H;1}(\kappa=0)}{M_{H;1}(\kappa=0)}\right)
  \exp(-\Gamma_1 h^{+-})J_1(\kappa r)\rm{d}\kappa \nonumber \\
  %
  &\quad - \frac{\zeta_1 (x^2-y^2)}{4\pi\gamma_1 r^4}
  \frac{R^-_{H;1}(\kappa=0)}{M_{H;1}(\kappa=0)}
  \exp(-\gamma_1 R^{+-}) \ .
  \label{eq:bookte}
\end{align}
}

The equation is marked in \bkcol{blue} to make it clear that the symbols and
parameters in \hun and in \book are not exactly the same.

The difference between equations \ref{eq:hunte} and \ref{eq:bookte} is that the
first one contains non-physical contributions. These have opposite signs in
TM$^{+-}$ and TE$^{+-}$, and therefore cancel each other out. But if we want to
know the specific contributions from TM and TE we have to remove them. The
non-physical contributions only affect the $J_1$-integrals, and only for
$\kappa = 0$.

The following lists for all 8 cases the term that has to be removed, in the
notation of \book (for the notation as in \hun see the implementation in
\tmtemod):
%
\bkcol{
\begin{align}
  TE^{++} &= + \frac{\zeta_1 (x^2-y^2)}{4\pi\gamma_1 r^4}
  \frac{\exp(-\gamma_1 |h^-|) }{M_{H;1}(\kappa=0)} \ ,
  \label{eq:1}\\
  %
  TE^{-+} &= - \frac{\zeta_1 (x^2-y^2)}{4\pi\gamma_1 r^4}
  \frac{R^+_{H;1}(\kappa=0)\exp(-\gamma_1 h^{-+}) }{M_{H;1}(\kappa=0)}\ , 
  \label{eq:2}\\
  %
  TE^{+-} &= - \frac{\zeta_1 (x^2-y^2)}{4\pi\gamma_1 r^4}
  \frac{R^-_{H;1}(\kappa=0)\exp(-\gamma_1 h^{+-}) }{M_{H;1}(\kappa=0)}\ , 
  \label{eq:3}\\
  %
  TE^{--} &= + \frac{\zeta_1 (x^2-y^2)}{4\pi\gamma_1 r^4}
  \frac{R^+_{H;1}(\kappa=0)R^-_{H;1}(\kappa=0)\exp(-\gamma_1 h^{--}) }
  {M_{H;1}(\kappa=0)}\ , 
  \label{eq:4}\\
  %
  %
  TM^{++} &= - \frac{\zeta_1 (x^2-y^2)}{4\pi\gamma_1 r^4}
  \frac{\exp(-\gamma_1 |h^-|) }{M_{V;1}(\kappa=0)}\ , 
  \label{eq:5}\\
  %
  TM^{-+} &= - \frac{\zeta_1 (x^2-y^2)}{4\pi\gamma_1 r^4}
  \frac{R^+_{V;1}(\kappa=0)\exp(-\gamma_1 h^{-+}) }{M_{V;1}(\kappa=0)}\ , 
  \label{eq:6}\\
  %
  TM^{+-} &= - \frac{\zeta_1 (x^2-y^2)}{4\pi\gamma_1 r^4}
  \frac{R^-_{V;1}(\kappa=0)\exp(-\gamma_1 h^{+-}) }{M_{V;1}(\kappa=0)}\ , 
  \label{eq:7}\\
  %
  TM^{--} &= - \frac{\zeta_1 (x^2-y^2)}{4\pi\gamma_1 r^4}
  \frac{R^+_{V;1}(\kappa=0)R^-_{V;1}(\kappa=0)\exp(-\gamma_1 h^{--}) }
  {M_{V;1}(\kappa=0)} \ .
  \label{eq:8}
\end{align}
}

Note that in equations \ref{eq:1} and \ref{eq:4} the correction terms have
opposite sign as those in equations \ref{eq:5} and \ref{eq:8} because at
$\kappa=0$ the TM and TE mode correction terms are equal. Also note that in
equations \ref{eq:2} and \ref{eq:3} the correction terms have the same sign as
those in equations \ref{eq:6} and \ref{eq:7} because at $\kappa=0$ the TM and
TE mode reflection responses in those terms are equal but with opposite sign:
\bkcol{$R^\pm_{V;1}(\kappa=0) = -R^\pm_{V;1}(\kappa=0)$}. 

\hun uses $\phi$, whereas \book uses $x$, $y$, for which we can use
%
\begin{equation}
  \cos(2\phi) = -\frac{x^2-y^2}{r^2} \ .
  \label{eq:phixy}
\end{equation}


% REFERENCES
\bibliographystyle{seg}
\begin{thebibliography}{}
\itemsep0pt

\bibitem[Hunziker et~al., 2015]{GEO.15.Hunziker}
Hunziker, J., J. Thorbecke, and E. Slob, 2015, The electromagnetic response in
  a layered vertical transverse isotropic medium: A new look at an old problem:
  Geophysics, {\bf 80}, F1--F18, doi: 10.1190/geo2013-0411.1.

\bibitem[Ziolkowski and Slob, 2017]{CUP.17.Ziolkowski}
Ziolkowski, A. and E. Slob, 2017, CSEM book (TODO, add citation once the book
  is out): Cambridge University Press.  ISBN: ??????.

\end{thebibliography}

\end{document}

